\ssr{ВВЕДЕНИЕ}

В настоящее время ASCII-графика используется в текстовых интерфейсах, таких как командные оболочки и системы без графического интерфейса, в качестве средства визуального представления данных \cite{TUI}, а также в художественных формах и интернет-коммуникации, где текстовые символы служат основным средством передачи визуальной информации. \cite{Future}

Целью данной работы является разработка программного обеспечения для построения трёхмерного изображения объёмных объектов и эллипсоидов с использованием заданного пользователем набора ASCII-символов. Пользователь также определяет количество, яркость и расположение белых точечных источников света и набор доступных символов.

Для достижения поставленной цели требуется решить следующие задачи:

1. Проанализировать существующие методы построения трехмерных объектов и преобразования изображений в ASCII-графику.

2. Спроектировать программное обеспечение для построения трёхмерного изображения.

3. Выбрать средства реализации и реализовать спроектированное программное обеспечение.

4. Исследовать зависимость качества изображения от выбранного набора символов.

Целевой областью применения разрабатываемого в данной работе программного обеспечения будут являться командные оболочки и системы без графического интерфейса, такие как серверы. Данные системы обладают ограниченными вычислительными возможностями.

\clearpage
